\documentclass[letter]{res}

\setlength{\textheight}{9.5in}
%\usepackage{fontspec}
\usepackage{geometry}
\usepackage{xcolor}
\usepackage[utf8]{inputenc}
\geometry{
 a4paper,
 total={210mm,297mm},
 left=20mm,
 right=25mm,
 top=17mm,
 bottom=17mm,
 bindingoffset=0mm
 }\usepackage{fontenc}
\usepackage{enumitem}
\setlist{nolistsep}
\usepackage[colorinlistoftodos]{todonotes}
\usepackage{tabu}
%\setmainfont[Mapping=tex-text]{Linux Libertine O}


\begin{document}

\name{Julian Mentasti Meza\\[14pt]}
\address{ (778)-882-3453 \\}
\address{
  \texttt{julian@mentasti.net}\\ } 
\begin{resume}
  \noindent\makebox[\linewidth]{\rule{\paperwidth}{0.4pt}}

\section{Skills}
{\sl Programming Languages:}  Java, JavaScript, Golang, Python, C++ and Racket\\
{\sl Markup Languages:} XML, HTML 5 and CSS \\
{\sl Database Management Systems:} MySQL, PostrgreSQL and SQLite \\
{\sl Operating Systems:} Linux/Unix systems (ArchLinux, Ubuntu, EdgeOS, CentOS) and Windows \\ 
{\sl Frameworks, Libraries and tools:} Django, Jupyter, Pandas, Git, OpenCV3, gRPC, XML-RPC and  \LaTeX \\ 
{\sl Languages:} Full fluency in English and Spanish \\
github: https://github.com/Julian-Mentasti
 \vspace{-4mm}

\section{Experience}
\textbf{UBC Orbit - Satellite Design Team}| Vancouver, CA\newline 
 - {\sl Payload Developer and Co-Captain} \hfill January 2018 – Present\\
 Worked to deploy a remote sensing satellite capable of identifying wildfires and forest loss by using machine learning models for the Canadian Satellite Design Challenge.\\
 \vspace{-2mm}
 \begin{itemize}
 \item Deployed a linear deblurring algorithm using the Lucy-Richardson algorithm.
 \item Created a Forrest fire identifier by inverting spectral bands and using haar cascades. 
 \item Created a system status service to monitor Payload's computing system. 
 \end{itemize}
 Currently, designing a new CubeSat that is is capable of taking a picture of earth and sending it back to any ham radio operator, as well as collaborating with teams from Simon Frasier University and University of Victoria to build OCRAASAT, a satellite capable of calibrating ground observatories such as CHIME. (https://www.ubcorbit.com)

\textbf{University of British Columbia - Computer Science Department}| Vancouver, CA \newline 
 - {\sl Undergraduate Teaching Assistant} \hfill September 2018 – Present\\
 Teaching Assistant for CPSC 110 - Computation, Programs and Programming. A first-year introductory programming course designed by Gregor Kiczales using Racket.

\textbf{CloudPBX inc}| Vancouver, CA \newline 
 - {\sl Software Developer intern} \hfill June 2018 – September 2018\\
Worked to enable their network research team by building tools that would improve and increase their data collection methods. \\
 \vspace{-2mm}
 \begin{itemize}
 \item Deployed a VPN tunnel that can connect various router models, network structures and bridge the flow of large amounts of data.
 \item Designed a web UI that allowed stakeholders to monitor, test and view the data from their routers.
 \item Implemented an auto installer script that setups the company's software on routers across different architectures and connects each device to the UI and the VPN network.
 \item Built a remote procedure call service monitor that runs on a router, acting as a watchdog for the firm's custom software and can orchestrate each piece of software accordingly.
 \end{itemize}

\section{Coursework {\sl(* will be completed by May 2018) }} \hfill
\vspace{-2mm}
\begin{tabu} { | X[l] | X[l] |}
 \hline
 *CPSC 425 - Computer Vision & CPSC 221 - Algorithms and Data Structures \\
 \hline
 *CPSC 320 - Intermediate Algorithm Design and Analysis  & CPSC 210 - Software Construction  \\
 \hline
 *CPSC 314 - Computer Graphics & CPSC 121 - Models of Computation \\
 \hline
 *CPSC 313 - Computer Hardware and Operating Systems & CPSC 110 - Computation, Programs and Programming \\
   \hline
 *STAT 302 - Probability & *CPSC 213 - Introduction to Computer Systems \\
  \hline
 Math 221 - Matrix Algebra & *Math 342 - Algebra and Coding Theory \\
\hline
\end{tabu}
 \vspace{-2mm}
\section{Education}
{\sl Bachelors of Science} - Combined Major in Computer Science and Statistics \hfill Expected: May 2021\\
University of British Columbia,  Vancouver, CA \hfill (GPA: 81/100)
  \vspace{-4mm}
  
International Baccalaureate Bilingual Diploma \hfill August 2015 – June 2017\\
Greengates School, Mexico City, MX
 \vspace{-2mm}

\section{Projects}
PIMS BC Data Science Workshop \newline - {\sl Individual} \\
I took part in a one week project, where as part of a team we would attempt to solve a problem for a firm using data science. I was part of the CloudPBX team, where we ended up cleaning, analyzing and displaying large sets of data. Our focus was centered around VOIP: how it worked and how we could better contextualize our data to suit the firms needs. 
\\
\begin{itemize}
 \item Created a new metric that was more sensitive to disturbances in a phone call.
 \item Discovered best and worst performing ISPs
 \item Grouped calls to find the highest caller density
\end{itemize}
http://workshop.bcdata.ca/2018/post/5-cloudpbx-project/

\vspace{-2mm}
\section{Activities} 
 UBC Quiz bowl | VP of Content (Present) \\
 UBC Scientific Software Seminar | Attendee (Present) \\
 
\vspace{-4mm}
\section{Awards} 
 Awarded Dean’s Honour List - 2018 UBC \\
 Creativity, Activity Service Award - 2017 Greengates \\
 IB Computer Science excellence Award - 2017 Greengates \\
 IB Business and Management excellence Award - 2017 Greengates \\

\end{resume}
\end{document}



